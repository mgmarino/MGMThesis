
\acknowledgments{
There are but few moments in one's life when one can truly ponder the road that
has lead to that point.  Looking back one can choose to wonder at the failures
narrowly averted, but how much more meaningful to realize the gracious helping
hands that made possible the way.  I devote this section to the latter, in the
hopes of conveying my gratitude to those from whom I benefitted so greatly.

My advisor and mentor, John Wilkerson, I thank for the support, the
encouragement, the living example, and the sometimes prodding it took to help
mold me into a scientist.  The experience as his student has given me
invaluable resources from which to draw as I embark upon my career.  I look
forward to any collaborative opportunity the future might hold. 

From the University of Notre Dame, I must thank the late Larry Lamm for
introducing me to the nitty gritty of experimental physics and for helping me
to realize I could contribute to the field.  My career as a physicist would
have likely ended after my undergraduate studies had it not been for Alejandro
Garcia, who fostered me through the earliest moments of my research career in
his previous post at Notre Dame. 

Juan Collar provided the brilliance and passion which made most of my thesis
work possible, and for that I am grateful.  Working with Juan there is truly never
a dull moment.  As well I must thank the staff at 
Soudan Underground Laboratory, including Jim Beaty, Dave Saranen, Jerry Meier, 
and Curt Lerol among others who made working underground a real joy. 

% Hamish, Leslie?

I must thank Michael Miller for being an advisor to matters-not-only-physics and
for being a friend.  Among many other things, he taught me that 
sometimes the most valuable thing to do to ensure research progress is to go skiing. 

The staff at CENPA has been so instrumental in supporting my time as a graduate
student.  I thank especially the front office, Victoria Clarkson and Kate
Higgins, for tackling the inevitable administrative issues I had and for
helping my life at the lab run so smoothly.  Dick Seymour was always eager to
help me solve a hard-to-find hardware or software issue.  His replacement, Gary
Holman, has already demonstrated his great value to the lab and I am
disappointed to have overlapped with him for only so short a time.  To Mark Howe
I am indebted for the always interesting discussions on hardware, software, or whatever, 
though I greatly missed their regularity after his move to North Carolina.    
To others at CENPA, I am grateful for the collegial atmosphere they generated
that made it truly a special place.

Jason Detwiler, my first postdoc mentor, I cannot thank enough for providing
endless resources and guidance.  His selflessness in helping others and me in
particular speaks to his character, skill and ability, and his example
challenges me to do likewise as I transition to a postdoctoral position. 

I am so lucky to have shared much of the past six years with my officemates,
Alexis Schubert and Rob Johnson.  It is difficult to summarize the office
rapport, but I will not forget the help and frank discussions that formed
much-anticipated interruptions to the work day.  I know that Alexis will continue
quietly crafting beautiful work and that Rob's memory will keep rivaling that
of an elephant. 

I thank the \MJ~collaboration, especially Reyco Henning, John Orrell, and David
Radford for aiding my development as a physicist by casting a critical eye to
my work. 

For my friends with whom I lived, played soccer, enjoyed bike rides, played
late night board games and took advantage of the outdoors (they know who they are): I am grateful for
the necessary balance that they brought to my life.

Finally, I must thank my family without whom this would not have been possible: to my
grandparents for giving me the gift of education and to my parents who taught
me the meaning of values and the value of meaning.  And to Verena, meine
\"{O}sterreicherin, I thank not only for having been on this road with me, but
also for being my companion as I walk ahead.  
% \vskip2pc {\narrower\noindent \par}
}


